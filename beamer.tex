\documentclass[xcolor=dvipsnames]{beamer}
% \documentclass[xcolor=dvipsnames,handout]{beamer}
% automatically loads:
% * ams{math, symb, text}
% * geometry
% * graphicx
% * xcolor
% * hyperref
% ...

% ## settings and commands
% # Encoding
\usepackage[utf8]{inputenc}  % allow utf-8 input
\usepackage[T1]{fontenc}  % use 8-bit T1 fonts

% # Control structures and utilities
\usepackage{calc}
\usepackage{ifthen}
\usepackage{xparse}  % more flexible macros with more than one optional argument

% # Document formatting
\usepackage{blindtext}  % similar to `lipsum`
\usepackage[activate={true,nocompatibility},final,factor=1100,stretch=10,shrink=10]{microtype}  % microtypography, improves general appearance
  % * activate - protrusion and expansion
  % * draft / final
  % * factor - add 10% to the protrusion amount (default is 1000)
  % * stretch / shrink - reduce stretchability/shrinkability (default is 20/20)
\usepackage{lipsum}  % Lorem ipsum
\usepackage{nicefrac}  % nicer fractions, (compact symbols for 1/2, etc.)
\usepackage{verbatim}
\usepackage{xspace}  % clever spacing for new commands (do not eat spaces)

% # Fonts
\usepackage{pifont}  % special PostScript fonts (e.g. for xmark == ding{53})

% # Bibliography
\usepackage{csquotes}
\usepackage[english]{babel}  % set default language to English

  % ## `apacite`
  \usepackage[natbibapa]{apacite}
    \bibliographystyle{apacite}

% # Tables
\usepackage{booktabs}  % professional-quality, elegant tables
\usepackage{colortbl}  % add color to LaTeX tables
% \usepackage{dcolumn}  % columns of different types and alignment
% \usepackage{longtable}  % enable pagebreaks in a table
% \usepackage{lscape}  % allow landscape mode
% \usepackage{multirow}  % tabular cells spanning multiple rows
\usepackage{tabularx}
\usepackage{tabu}  % new interface for tables

% # Math
% ## AMS facilities
\usepackage{amsmath}
\usepackage{amsthm}
\usepackage{amsfonts}  % blackboard math symbols
\usepackage{amssymb}
  \allowdisplaybreaks

% ## other math fonts
\usepackage{bbm}
\usepackage{latexsym}
\usepackage{mathrsfs}  % \mathscr
% \usepackage{yhmath}  % extended maths fonts for LaTeX

% ## additional tools
\usepackage[makeroom]{cancel}
% \usepackage{commath}  % math delimiters of auto computed size
\usepackage[retainorgcmds]{IEEEtrantools}  % sophisticated equation arrays
  \renewcommand*{\IEEEeqnarraydecl}{%
    \setlength{\jot}{2\IEEEnormaljot}% twice the normal value of \jot
  }
% \usepackage{kbordermatrix}  % add column and row labels to matrices
\usepackage{mathtools}  % \mathclap ...
\usepackage{physics}  % \bra, \ket, \expval, \mel ...
                      % & \abs, \norm, \var, \tr, \Tr ...

% # Graphics
\usepackage{caption}
\usepackage{float}  % improved interface for floating objects
\usepackage{movie15}  % inline figures
\usepackage{subcaption}  % subfigures
% \usepackage{graphicx}  % enhanced support for graphics
  \graphicspath{{figures/}{../figures/}}  % in gereral {{subdir1/}...{subdirn/}}

% # Colors
% \usepackage[dvipsnames]{xcolor}  % color (+ in math mode)
  % pre-defined colors: https://en.wikibooks.org/wiki/LaTeX/Colors#Predefined_colors

% ## TUM colors
\definecolor{Pantone300C}{HTML}{0065BD}  % TUM primary blue
% \definecolor{Pantone301}{HTML}{005293}   % TUM secondary light blue
% \definecolor{Pantone540}{HTML}{003359}   % TUM secondary dark blue
% \definecolor{DarkGray}{HTML}{333333}     % TUM secondary dark gray
\definecolor{MediumGray}{HTML}{808080}   % TUM secondary medium gray
\definecolor{LightGray}{HTML}{CCCCC6}    % TUM secondary light gray
\definecolor{Pantone7527}{HTML}{DAD7CB}  % TUM accent gray
\definecolor{Pantone158}{HTML}{E37222}   % TUM accent orange
\definecolor{Pantone383}{HTML}{A2AD00}   % TUM accent green
\definecolor{Pantone283}{HTML}{98C6EA}   % TUM accent very light blue
\definecolor{Pantone542}{HTML}{64A0C8}   % TUM accent light blue

% ## my colors
\definecolor{tumblue}{HTML}{0065BD}
\definecolor{darkblue}{HTML}{001473}
\definecolor{darkgray}{HTML}{646464}
\definecolor{ddarkgray}{HTML}{444444}
% \definecolor{gray}{rgb}{0.5,0.5,0.5}
\definecolor{orange}{rgb}{1.0,0.3,0.01}
\definecolor{violet}{rgb}{0.4,0.0,0.6}
\definecolor{yellow}{rgb}{1.0,0.7,0.0}
\definecolor{boxc}{rgb}{1.0, 1.0, .9}
\definecolor{mygreen}{rgb}{0, 0.75, 0.18}

% ## define colors for `hyperref`
\def\linkcolor{tumblue}
\def\urlcolor{darkblue}
\def\citecolor{darkblue}

% # Algorithms (pseudocode)
\usepackage{algorithm,setspace}
\usepackage{algorithmicx}
\usepackage[noend]{algpseudocode}
  % \algrenewcommand{\algorithmiccomment}[1]{\hfill$\rightarrow #1}  % normal arrow comments
  \newcommand{\commentsymbol}{//}% or \% or $\triangleright$
  \algrenewcommand\algorithmiccomment[1]{\hfill \commentsymbol{} \textcolor{darkgray}{#1}}
  \makeatletter
  \newcommand{\LineComment}[2][\algorithmicindent]{\Statex{}\hspace{#1}\commentsymbol{} \textcolor{darkgray}{#2}}
  \makeatother
  \newcommand{\varfont}{\texttt}

% # Hyperlinks
% \usepackage[linktocpage,colorlinks=true,bookmarksnumbered=true]{hyperref}  % hyperlinks

% ## PDF metadata
\hypersetup{
  pdftitle={The Title},%
  pdfauthor={The Author},%
  pdfcreator={The Creator},%
  pdfproducer={The Producer},%
  pdfkeywords={{Keyword1},{Keyword2},{Keyword3}},%
  pdfsubject={The Subject},
}

% # General LaTeX shortcuts
\renewcommand{\u}[1]{\underline{#1}}


% # Custom symbols
\newcommand{\gooditem}{\item[\checkmark]}
\newcommand{\baditem}{\item[\ding{53}]}
\newcommand{\good}{\textbf{\color{green}[\checkmark]}}
\newcommand{\bad}{\textbf{\color{red}[\ding{53}]}}


% # Graphics
% ## insert figure (fname, caption for ToC, caption, label, width as fraction of \textwidth)
\newcommand{\insertfig}[5]{
	\begin{figure}[htbp]
		\begin{center}
			\includegraphics[width=#5\textwidth]{#1}  % width, height, scale, angle ...
		\end{center}
		\vspace{-0.4cm}
		\caption[#2]{#3}
		\label{fig:#4}
	\end{figure}
}

\input{preamble/math}
% ## Semantic formatting
\newcommand{\alg}[1]{\textsc{#1}}  % sf, tt
\newcommand{\cmd}[1]{\texttt{#1}}

\newcommand{\s}[1]{\ensuremath{#1}}  % scalar
\newcommand{\w}[1]{\ensuremath{\mathbf{#1}}}  % vector
\newcommand{\bs}[1]{\ensuremath{\boldsymbol{#1}}}  % vector
\newcommand{\m}[1]{\ensuremath{\boldsymbol{#1}}}  % matrix
% sans serif -- tensor
\newcommand{\g}[1]{\textsc{#1}}  % classical gate
\newcommand{\q}[1]{\ensuremath{\boldsymbol{\mathsf{#1}}}}  % quantum gate
% \newcommand{\dist}[1]{\ensuremath{#1}}  % distribution

\renewcommand{\c}[1]{\ensuremath{\mathcal{#1}}}  % concept

% data-spaces \mathcal{X}, \mathcal{Y}; data \mathcal{D}
% \newcommand{\Pth}{\mathrm{P}_{\theta}}
% \newcommand{\Pz}{\mathrm{P}_z}

% BM notation
\renewcommand{\v}{\w{v}}
\newcommand{\h}{\w{h}}
\newcommand{\x}{\w{x}}
\newcommand{\y}{\w{y}}
\newcommand{\z}{\w{z}}

\newcommand{\W}{\m{W}}
\renewcommand{\L}{\m{L}}
\newcommand{\J}{\m{J}}
\newcommand{\gen}{\ensuremath{\bs{\theta}}}
\newcommand{\F}{\mathcal{F}}

% Q notation
\renewcommand{\a}{\ensuremath{\bs{\alpha}}}


% ## theme
% \usetheme{Antibes}
% \usetheme{Warsaw}
% \usetheme{Berlin}
% \usetheme{Darmstadt}
% \usetheme{Ilmenau}
\usetheme{Madrid}

% ## outer theme
\useoutertheme{default}
% \useoutertheme{infolines}
% \useoutertheme{smoothbars}
% \useoutertheme{smoothtree}
% [x] \useoutertheme{miniframes}
% [x] \useoutertheme{sidebar}
% [x] \useoutertheme{split}
% [x] \useoutertheme{shadow}
% [x] \useoutertheme{tree}

% ## inner theme
% \useinnertheme{default}
\useinnertheme{circles}
% [x] \useinnertheme{rectangles}
% [x] \useinnertheme{rounded}
% [x] \useinnertheme{inmargin}

% ## color theme
% \usecolortheme[named=Maroon]{structure}
% \usecolortheme[named=ForestGreen]{structure}
% \usecolortheme[named=RoyalPurple]{structure}
% \usecolortheme[named=ddarkgray]{structure}
% \usecolortheme[named=Mahogany]{structure}
\definecolor{UBCblue}{rgb}{0.04706, 0.13725, 0.26667} % UBC Blue (primary)
\definecolor{dddarkgrey}{rgb}{0.15, 0.15, 0.15}
\definecolor{darkpersianindigo}{rgb}{0.12 , 0.042, 0.288}
% \usecolortheme[named=UBCblue]{structure}
\usecolortheme[named=dddarkgrey]{structure}
% \usecolortheme[named=darkpersianindigo]{structure}

% ## font theme
% \usefonttheme{default}
% \usefonttheme{serif}
\usefonttheme{professionalfonts}
% \usefonttheme{structurebold}
% \usefonttheme{structureitalicserif}
% \usefonttheme{structuresmallcapsserif}

% ## title
\title{The Tutorial}
\titlegraphic{\includegraphics[width=0.2\textwidth]{death-star}}
\subtitle{An Introduction}
\author{The Author}

% # main
\begin{document}

  % ## highlight the upcoming section in ToC
  \AtBeginSection[]{
    \begin{frame}
      \frametitle{Outline}
      \tableofcontents[currentsection,hidesubsections]
    \end{frame}
  }

  % ## highlight the upcoming subsection in ToC
  \AtBeginSubsection[]{
    \begin{frame}
      \frametitle{Outline}
      \tableofcontents[currentsubsection]
    \end{frame}
  }

  % ## title
  \begin{frame}
    \titlepage{}
  \end{frame}

  % ## initial ToC
  \begin{frame}{Outline}
    \tableofcontents[hideallsubsections]
  \end{frame}

  % ## sections
  \section{Introduction}

  \begin{frame}[fragile]{test}
    \begin{definition}
      some text \emph{emph} as {\color{blue}transfer} as.
      \\
      \alert{alert} a \textbf{R} a.
    \end{definition}
    %
    \pause
    %
    \begin{exampleblock}{Nice}
      \begin{IEEEeqnarray}{rCl}
        \binom{n}{k}
          &=& \frac{n!}{k!(n-k)!} \label{eq:label1}
          \\[0.5em]
          &=& \frac{1}{2\pi i}\oint\limits_{\Gamma} \frac{ (1+z)^n }{z^{k+1}} \ud z
          \label{eq:label2}
      \end{IEEEeqnarray}
    \end{exampleblock}
  \end{frame}

  \begin{frame}
    \begin{gather}
      \textup{\mathsc{RMSProp}} \; \alg{Adam} \; \cmd{pmatrix} \\
      \s{a} \; \w{x} \; \bs{\alpha} \; \m{A} \; \q{A} \\
      \g{not}\text{ gate} \; \q{CNOT}\text{ gate} \\
      \c{X} \; \c{Y} \; \c{D}
    \end{gather}
  \end{frame}

  \begin{frame}{test2}
    \begin{itemize}
      \item $\var X \leq (M - \E X)(\E X - m)$
      \pause
      \item $\KL\p{P_{\text{data}} \parallel P_{\text{model}}}$
      \pause
      \item
      \begin{equation}
        \text{Normal}(\mathrm{x} \mid \mu, \sigma^2 )
      \end{equation}
    \end{itemize}
  \end{frame}

  \begin{frame}{test2}
    \begin{enumerate}
      \item<1-> \alt<2> {\color{blue} A}{\color{grey} a}
      \item<1-> \alt<3> {\color{blue} B}{\color{grey} b}
      \item<1-> \alt<4> {\color{blue} C}{\color{grey} c}
    \end{enumerate}
  \end{frame}

  \begin{frame}{test3}
    \begin{table}[h]
      \centering
      \caption{Caption}
      \label{tab:label}
      \begin{tabular}{cc}
        \toprule
        A & $\mathbb{R}$ \\
        \midrule
        \onslide<2-> a & b \\
        \onslide<3-> c & d \\
        \midrule
        \onslide<4-> e & f \\
        \onslide<5-> g & h \\
        \bottomrule
      \end{tabular}
      % OR caption & label here ...
    \end{table}
  \end{frame}

  \begin{frame}{test4}
    \begin{center}
      \begin{tabular}{ccc}
        \rowcolor{blue!50} a & b \\
        \rowcolor{blue!30} a & b \\
        \rowcolor{blue!10} a & b \\
        \rowcolor{blue!30} a & b \\
        \rowcolor{blue!10} a & b
      \end{tabular}
    \end{center}
  \end{frame}

  \section{Graphics}

  \begin{frame}{Includegraphics}
    \begin{center}
      \includegraphics[height=0.5\textheight]{death-star}
    \end{center}
  \end{frame}

  \begin{frame}{Insertfig}
    \insertfig{death-star}{ds-toc}{Death Star}{ds}{0.25}
  \end{frame}

  \begin{frame}{Subfigures}
    \begin{figure}[htb]
      \centering
      %
      \begin{subfigure}[b]{.250\textwidth}
        \frame{\includegraphics[width=\textwidth]{death-star}}
        \caption{Caption 1}  % or \subcaption
        \label{fig:1}
      \end{subfigure}
      % \hfill
      \qquad
      \qquad
      \begin{subfigure}[b]{.250\textwidth}
        \includegraphics[width=\textwidth]{death-star}
        \caption{Caption 2}
        \label{fig:2}
      \end{subfigure}
      \\
      \begin{subfigure}[b]{.175\textwidth}
        \frame{\includegraphics[width=\textwidth]{death-star}}
        \caption{Caption 3}
        \label{fig:3}
      \end{subfigure}
      \qquad
      \begin{subfigure}[b]{.175\textwidth}
        \includegraphics[width=\textwidth]{death-star}
        \caption{Caption 4}
        \label{fig:4}
      \end{subfigure}
      \qquad
      \begin{subfigure}[b]{.175\textwidth}
        \includegraphics[width=\textwidth]{death-star}
        \caption{Caption 5}
        \label{fig:5}
      \end{subfigure}
      %
      \caption{The caption. \emph{Top}: top. \emph{Bottom}: bottom.}
      \label{fig:subfigures}
    \end{figure}
  \end{frame}

  \section{Examples}

  \subsection{Example A}

  \begin{frame}{slide}
  \end{frame}

  \subsection{Example B}

  \begin{frame}{slide}
    \citep{tf}
  \end{frame}

  \begin{frame}{References}
    \bibliography{bibliography/bib}
  \end{frame}


\end{document}
