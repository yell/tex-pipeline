% -----------------------------
% ----- General shortcuts -----
% \renewcommand{\sf}[1]{\textsf{#1}}
% \renewcommand{\tt}[1]{\texttt{#1}}
% \renewcommand{\bf}[1]{\textbf{#1}}
% \renewcommand{\sl}[1]{\textsl{#1}}
% \renewcommand{\sc}[1]{\textsc{#1}}
% \newcommand{\e}[1]{\emph{#1}}

\renewcommand{\u}[1]{\underline{#1}}

\newcommand{\gooditem}{\item[\checkmark]}
\newcommand{\baditem}{\item[\ding{53}]}
\newcommand{\good}{\textbf{\color{green}[\checkmark]}}
\newcommand{\bad}{\textbf{\color{red}[\ding{53}]}}


% ----------------------------
% ----- General commands -----
\newcommand{\commentinline}[1]{\color{darkgray}\left[\text{#1}\right]\color{black}}
\newcommand{\commentii}[2]{\color{darkgray}\left[\begin{array}{c}\text{#1}\\ \text{#2}\end{array}\right]\color{black}}


% --------------------------------------
% ----- Document layout/formatting -----

% page clearing
\newcommand{\clearemptydoublepage}{%
  \ifthenelse{\boolean{@twoside}}{\newpage{\pagestyle{empty}\cleardoublepage}}%
  {\clearpage}}

% comment that appears on the border
\newcommand{\mcomment}[1]{\marginpar{\raggedright \noindent {\textsl{#1}}}}


% insert figure (fname, caption for ToC, caption, label, width as fraction of \textwidth)
\newcommand{\insertfig}[5]{
	\begin{figure}[htbp]
		\begin{center}
			\includegraphics[width=#5\textwidth]{#1}  % width, height, scale, angle ...
		\end{center}
		\vspace{-0.4cm}
		\caption[#2]{#3}
		\label{fig:#4}
	\end{figure}
}

% --- color box ---
\newlength\mytemplen
\newsavebox\mytempbox

\makeatletter

\newcommand\mybox{%
    \@ifnextchar[%]
       {\@mybox}%
       {\@mybox[0pt]}}

\def\@mybox[#1]{%
    \@ifnextchar[%]
       {\@@mybox[#1]}%
       {\@@mybox[#1][0pt]}}

\def\@@mybox[#1][#2]#3{
    \sbox\mytempbox{#3}%
    \mytemplen\ht\mytempbox
    \advance\mytemplen #1\relax
    \ht\mytempbox\mytemplen
    \mytemplen\dp\mytempbox
    \advance\mytemplen #2\relax
    \dp\mytempbox\mytemplen
    \colorbox{boxc}{\hspace{0em}\usebox{\mytempbox}\hspace{0em}}}

\makeatother


% ----------------------
% ----- References -----

% suffixes are consistent with:
% https://en.wikibooks.org/wiki/LaTeX/Labels_and_Cross-referencing
\newcommand{\refch}[1]{chapter \ref{ch:#1}}
\newcommand{\refsec}[1]{section \ref{sec:#1}}
\newcommand{\refsubsec}[1]{subsection \ref{subsec:#1}}
\newcommand{\reffig}[1]{figure \ref{fig:#1}}
\newcommand{\reftab}[1]{table \ref{tab:#1}}
\renewcommand{\refeq}[1]{equation \eqref{eq:#1}}
\newcommand{\reflst}[1]{listing \ref{lst:#1}}  % code listing
\newcommand{\refitm}[1]{\ref{itm:#1}}  % enumerated list item
\newcommand{\refalg}[1]{algorithm \ref{alg:#1}}
\newcommand{\refapp}[1]{appendix \ref{app:#1}}

\newcommand{\refCh}[1]{Chapter \ref{ch:#1}}
\newcommand{\refSec}[1]{Section \ref{sec:#1}}
\newcommand{\refSubsec}[1]{Subsection \ref{subsec:#1}}
\newcommand{\refFig}[1]{Figure \ref{fig:#1}}
\newcommand{\refTab}[1]{Table \ref{tab:#1}}
\newcommand{\refEq}[1]{Equation \eqref{eq:#1}}
\newcommand{\refLst}[1]{Listing \ref{lst:#1}}  % code listing

\newcommand{\refAlg}[1]{Algorithm \ref{alg:#1}}
\newcommand{\refApp}[1]{Appendix \ref{app:#1}}


% --------------------------------------------
% ----- Theorems environments & counters -----
% \newtheorem{theorem}{Theorem}[chapter]
% \newtheorem{lemma}[theorem]{Lemma}
% \newtheorem{corollary}[theorem]{Corollary}
% \newtheorem{remark}[theorem]{Remark}
% \newtheorem{definition}[theorem]{Definition}
% \newtheorem{equat}[theorem]{Equation}
% \newtheorem{example}[theorem]{Example}
% \newtheorem{algorithm}[theorem]{Algorithm}


% ----------------------
% ----- Math stuff -----

% --- general shortcuts ---
\renewcommand{\l}{\left}
\renewcommand{\r}{\right}
\renewcommand{\o}[1]{\overline{#1}}
\newcommand{\ub}[2]{\underbrace{#1}_{#2}}
\newcommand{\ob}[2]{\overbrace{#1}_{#2}}

% --- fonts & boldness ---
\newcommand{\bs}[1]{\boldsymbol{#1}}
\newcommand{\mr}[1]{\mathrm{#1}}
\newcommand{\mb}[1]{\mathbf{#1}}
\newcommand{\mc}[1]{\mathcal{#1}}
\newcommand{\mf}[1]{\mathfrak{#1}}

% --- shortcuts to common quantities ---
\newcommand{\half}{\frac{1}{2}}

% --- shortcuts for delimiters with correct auto-size ---
% (names are consistent with `matrix` environment)
\newcommand{\p}[1]{ \left( #1 \right) }
\renewcommand{\b}[1]{ \left[ #1 \right] }
\newcommand{\B}[1]{ \left\{ #1 \right\} }
\renewcommand{\v}[1]{ \left| #1 \right| }
\newcommand{\V}[1]{ \left\| #1 \right\| }

% --- general math ---
\DeclareMathOperator*{\argmax}{arg\,max}
\DeclareMathOperator*{\argmin}{arg\,min}
\DeclareMathOperator{\const}{const}
\DeclareMathOperator{\e}{e}  % exp

% better defaults
\renewcommand{\epsilon}{\varepsilon}
\newcommand{\eps}{\epsilon}
\renewcommand{\kappa}{\varkappa}
\renewcommand{\geq}{\geqslant}
\renewcommand{\leq}{\leqslant}

% spaces
\newcommand{\N}{\mathbb{N}}
\newcommand{\Z}{\mathbb{Z}}
\newcommand{\Q}{\mathbb{Q}}
\newcommand{\R}{\mathbb{R}}
\newcommand{\C}{\mathbb{C}}

% \DeclarePairedDelimiter{\abs}{\lvert}{\rvert}
\let\abs\relax\newcommand{\abs}[1]{\left\lvert #1 \right\rvert}

% --- linear algebra ---
\let\tr\relax\DeclareMathOperator{\tr}{tr}
\let\Tr\relax\DeclareMathOperator{\Tr}{Tr}

% \newcommand{\norm}[1]{\left\lVert #1 \right\rVert}
\newcommand{\vecii}[2]{\begin{array}[h]{c} #1 \\ #2	\end{array}}
\newcommand{\veciii}[3]{\begin{array}[h]{c} #1 \\ #2 \\ #3	\end{array}}
\newcommand{\veciv}[4]{\begin{array}[h]{c} #1 \\ #2 \\ #3 \\ #4	\end{array}}

\newcommand{\matii}[4]{\begin{array}[h]{ccc} #1 & #2 \\ #3 & #4 \end{array}}
\newcommand{\matiii}[9]{\begin{array}[h]{ccc} #1 & #2 & #3 \\ #4 & #5 & #6 \\ #7 & #8 & #9	\end{array}}

% --- calculus ---
\newcommand{\evalat}[1]{\biggr\rvert _{#1}}
% \newcommand{\pd}[2]{ \frac{\partial #1}{\partial #2} }  % partial derivative
\DeclareDocumentCommand{\pd}{ O{} O{} m }{\frac{\partial^{#2}#1}{\partial#3^{#2}}}  % partial derivative

\renewcommand{\d}{\mathrm{d}}  % upright "d" for differential
\newcommand{\ud}{\,\mathrm{d}}  % correct spacing for differential in integral

% --- probability theory ---
\let\var\relax\DeclareMathOperator{\var}{var}
\DeclareMathOperator{\med}{med}

\newcommand{\E}{\mathbb{E}}
\renewcommand{\P}[1]{\mathbb{P}\left\lbrace #1 \right\rbrace}
\renewcommand{\Pr}[1]{\operatorname{Pr}\left\lbrace #1 \right\rbrace}

% --- information theory ---
\newcommand{\KL}[2]{\operatorname{KL}(#1 \parallel #2)}
\newcommand{\DKL}[2]{D_{\operatorname{KL}}(#1 \parallel #2)}


% ----------------
% ----- Misc -----

% --- some abreviations ---
\newcommand{\Reg}{$^{\textregistered}$}
\newcommand{\reg}{$^{\textregistered}$ }
\newcommand{\Tm}{\texttrademark}
\newcommand{\tm}{\texttrademark~}
\newcommand{\bsl}{$\backslash$}
