% # Encoding
\usepackage[utf8]{inputenc} % allow utf-8 input
\usepackage[T1]{fontenc}    % use 8-bit T1 fonts


% # Control structures and utilities
\usepackage{calc}
\usepackage{ifthen}
% \usepackage{thumbpdf}  % make PDF thumbnail
\usepackage{xparse}  % more flexible macros with more than one optional argument


% # Document structure
\usepackage[acronym,symbols,nomain,nogroupskip,nonumberlist,nopostdot,toc]{glossaries}
  % * xindy - more flexible than makeindex (sudo apt install xindy)
\usepackage{glossary-longbooktabs}
  \newglossary[slg]{symbolslist}{syi}{syg}{Symbolslist}
  \makeglossaries
% \usepackage{makeidx}  % standard LaTeX package for creating indexes
\usepackage{subfiles}


% # Document formatting
\usepackage{a4wide}  % set the geometry to use a reasonable amount of width
\usepackage[titletoc]{appendix}  % prepend "Appendix" to chapter names
\usepackage{blindtext}  % similar to `lipsum`
\usepackage{empheq}  % for colorbox
\usepackage{enumitem}  % customizable lists, e.g. `noitemsep`
% \usepackage{fancyhdr}  % customizing the headers and footers
% \usepackage{fancyvrb}  % highly customisable verbatim
\usepackage[symbol]{footmisc}  % symbols for footnotemark instead of numbers
% \usepackage{geometry}  % page dimensions, margins ...
\usepackage{marginnote}  % margin notes
  \usepackage{mparhack}
  \usepackage{marginfix}
\usepackage[activate={true,nocompatibility},final,factor=1100,stretch=10,shrink=10]{microtype}  % microtypography, improves general appearance
  % * activate - protrusion and expansion
  % * draft / final
  % * factor - add 10% to the protrusion amount (default is 1000)
  % * stretch / shrink - reduce stretchability/shrinkability (default is 20/20)
\usepackage{lipsum}  % Lorem ipsum
\usepackage{mdframed}  % for box around floating bodies (figures, algorithms...)
% \usepackage{multicol}  % typeset text in multiple columns
\usepackage{nicefrac}  % nicer fractions, (compact symbols for 1/2, etc.)
% \usepackage{paralist}  % Improves enumerate and itemize. Also provides some compact environments
% \usepackage[parfill]{parskip}  % use no indentation and space between paragraphs
% \usepackage{quotchap}  % fancy chapter heading pages
\usepackage{rotating}  % rotation tools, including rotated full-page floats
% \usepackage{tcolorbox}  % color box
  % \tcbuselibrary{skins,breakable}
% \usepackage{thmtools}  % Theorem styling
\usepackage{titlesec}  % help with section naming
% \usepackage{tocloft}  % ToC formatting
\usepackage[hyphens,spaces]{url}  % simple URL typesetting
\usepackage{verbatim}
\usepackage{xspace}  % clever spacing for new commands (do not eat spaces)


% # Fonts
% \usepackage{dsfont}
% \usepackage[scaled=0.8]{luximono}  % fixed-width font that supports boldface (useful for typesetting source code)
% \usepackage[sc]{mathpazo}  % like `palatino` but not deprecated
% \usepackage{palatino}  % ugly font [deprecated]
\usepackage{pifont}  % special PostScript fonts (e.g. for xmark == ding{53})


% # Bibliography
% ## `biblatex`
  \usepackage{csquotes}
  \usepackage[english]{babel}  % set default language to English
% \usepackage[backend=biber,
%             bibstyle=authoryear,
%             citestyle=authoryear,%apa,
%             % natbib=true,
%             hyperref=true,
%             isbn=true,
%             doi=true,
%             url=true,
%             eprint=true]{biblatex}
  % \DeclareLanguageMapping{english}{english-apa}

% ## `apacite`
\usepackage[natbibapa]{apacite}
  \bibliographystyle{apacite}

% ## plain `natbib`
% \usepackage[round,sort]{natbib}
% \bibliographystyle{plainnat}

% ## utilities
\usepackage[nottoc,numbib]{tocbibind}  % add bibliography & listings to ToC


% # Tables
\usepackage{booktabs}  % professional-quality, elegant tables
% \usepackage{colortbl}  % add color to LaTeX tables
% \usepackage{dcolumn}  % columns of different types and alignment
% \usepackage{longtable}  % enable pagebreaks in a table
% \usepackage{lscape}  % allow landscape mode
% \usepackage{multirow}  % tabular cells spanning multiple rows
\usepackage{tabularx}
\usepackage{tabu}  % new interface for tables


% # Math
% ## AMS facilities
\usepackage{amsmath}
\usepackage{amsthm}
\usepackage{amsfonts}  % blackboard math symbols
\usepackage{amssymb}

% ## other math fonts
\usepackage{bbm}
\usepackage{latexsym}
\usepackage{mathrsfs}  % \mathscr
% \usepackage{yhmath}  % extended maths fonts for LaTeX

% ## additional tools
\usepackage[makeroom]{cancel}
% \usepackage{commath}  % math delimiters of auto computed size
\usepackage[retainorgcmds]{IEEEtrantools}  % sophisticated equation arrays
% \usepackage{kbordermatrix}  % add column and row labels to matrices
\usepackage{mathtools}  % \mathclap ...
\usepackage{physics}  % \bra, \ket, \expval, \mel ...
                      % & \abs, \norm, \var, \tr, \Tr ...


% # Graphics
\usepackage{caption}
\usepackage{float}  % improved interface for floating objects
\usepackage{graphicx}  % enhanced support for graphics
  \graphicspath{{figures/}{../figures/}}  % in gereral {{subdir1/}...{subdirn/}}
\usepackage{movie15}  % inline figures
% \usepackage{pgfplots}  % uses PGF to draw professionally looking charts and plots
\usepackage{subcaption}  % subfigures
% \usepackage{textpos}  % absolute positioning on the page
% \usepackage{tikz}  % `tikzpictures`
\usepackage{titlepic}  % add a picture to the title page (requires `titlepage` for articles)
% \usepackage{wrapfig}  % allow text to wrap around figures


% # Colors
\usepackage[dvipsnames]{xcolor}  % color (+ in math mode)
  % pre-defined colors: https://en.wikibooks.org/wiki/LaTeX/Colors#Predefined_colors

% ## my colors
\definecolor{tumblue}{HTML}{0065BD}
\definecolor{darkblue}{HTML}{001473}
\definecolor{darkgray}{RGB}{100, 100, 100}
\definecolor{ddarkgray}{RGB}{66, 66, 66}
\definecolor{gray}{rgb}{0.5,0.5,0.5}
\definecolor{orange}{rgb}{1.0,0.3,0.01}
\definecolor{violet}{rgb}{0.4,0.0,0.6}
\definecolor{yellow}{rgb}{1.0,0.7,0.0}
\definecolor{boxc}{rgb}{1.0, 1.0, .9}
\definecolor{mygreen}{rgb}{0, 0.75, 0.18}

% ## define colors for `hyperref`
\def\linkcolor{tumblue}
\def\urlcolor{darkblue}
\def\citecolor{darkblue}


% # Theorems environments & counters
% \newtheorem{theorem}{Theorem}[chapter]
% \newtheorem{lemma}[theorem]{Lemma}
% \newtheorem{corollary}[theorem]{Corollary}
% \newtheorem{remark}[theorem]{Remark}
% \newtheorem{definition}[theorem]{Definition}
% \newtheorem{equat}[theorem]{Equation}
% \newtheorem{example}[theorem]{Example}
% \newtheorem{algorithm}[theorem]{Algorithm}
% ---
% \newtheorem{thm}{Theorem}[section]
% \newtheorem{theorem}{Theorem}[section]
% \newtheorem{lemma}[theorem]{Lemma}
% \newtheorem{proposition}[theorem]{Proposition}
% \newtheorem{corollary}[theorem]{Corollary}
% \newtheorem{definition}[theorem]{Definition}


% # Listings
% \usepackage{listings}  % automatically colour code
% \usepackage[newfloat]{minted} % (recommended)
% Set global Minted options
% \setminted{linenos, autogobble, frame=lines, framesep=2mm}
%%% Inline C++ (optional)
% \newcommand{\incpp}[1]{\mintinline{c++}{#1}}
% \newenvironment{code}{\captionsetup{type=listing}}{}
% \SetupFloatingEnvironment{listing}{name=Source Code}

% ## python listing
% \floatstyle{plain}
% \restylefloat{figure}
% \lstset{frame=tb,
% 	language=Python,
% 	aboveskip=3mm,
% 	belowskip=3mm,
% 	showstringspaces=false,
% 	columns=flexible,
% 	basicstyle={\small\ttfamily},
% 	numbers=none,
% 	numberstyle=\color{violet},
% 	keywordstyle=\color{orange},
% 	commentstyle=\color{gray},
% 	stringstyle=\color{yellow},
% 	breaklines=false,
% 	breakatwhitespace=false,
% 	tabsize=4
% }

% ## Matlab listings
% \lstloadlanguages{Matlab}%
% \lstset{language=Matlab,                        % Use MATLAB
%         frame=single,                           % Single frame around code
%         basicstyle=\small\ttfamily,             % Use small true type font
%         keywordstyle=[1]\color{blue}\bf,        % MATLAB functions bold and blue
%         keywordstyle=[2]\color{purple},         % MATLAB function arguments purple
%         keywordstyle=[3]\color{blue}\underbar,  % User functions underlined and blue
%         identifierstyle=,                       % Nothing special about identifiers
%                                                 % Comments small dark green courier
%         commentstyle=\usefont{T1}{pcr}{m}{sl}\color{MyDarkGreen}\small,
%         stringstyle=\color{purple},             % Strings are purple
%         showstringspaces=false,                 % Don't put marks in string spaces
%         tabsize=3,                              % 5 spaces per tab
%         %
%         %%% Put standard MATLAB functions not included in the default
%         %%% language here
%         morekeywords={xlim,ylim,var,alpha,factorial,poissrnd,normpdf,normcdf},
%         %
%         %%% Put MATLAB function parameters here
%         morekeywords=[2]{on, off, interp},
%         %
%         %%% Put user defined functions here
%         morekeywords=[3]{FindESS, homework_example},
%         %
%         morecomment=[l][\color{blue}]{...},     % Line continuation (...) like blue comment
%         numbers=left,                           % Line numbers on left
%         firstnumber=1,                          % Line numbers start with line 1
%         numberstyle=\tiny\color{blue},          % Line numbers are blue
%         stepnumber=1                        % Line numbers go in steps of 5
%         }


% # Algorithms (pseudocode)
\usepackage{algorithmicx}
\usepackage{algpseudocode}
  \algrenewcommand{\algorithmiccomment}[1]{\hfill$\rightarrow$ #1}  % normal arrow comments


% # Hyperlinks
\usepackage[linktocpage,colorlinks=true,bookmarksnumbered=true]{hyperref}  % hyperlinks

% ## set links colors
\hypersetup{
  linkcolor=\linkcolor,%
  urlcolor=\urlcolor,%
  citecolor=\citecolor
}

% ## disable the coloring of the links when printing
\usepackage[ocgcolorlinks]{ocgx2}[2017/03/30]

% ## PDF metadata
\hypersetup{
  pdftitle={The Title},%
  pdfauthor={The Author},%
  pdfcreator={The Creator},%
  pdfproducer={The Producer},%
  pdfkeywords={{Keyword1},{Keyword2},{Keyword3}},%
  pdfsubject={The Subject},
}


% # Dependent packages (on `hyperref`, `xcolor`, `url` ...)
\usepackage{bookmark}  % better bookmark handling (loads hyperref)
\usepackage{hyperxmp}  % create XMP Metadata (uses the values from hyperref)
\usepackage{todonotes}  % \todo, \listoftodos
